\selectlanguage{english}
\chapter*{Control a Robot via VEP Using Emotiv EPOC}

\section*{Abstract}

\normalsize
This thesis describes an \acrshort{SSVEP}-based \acrshort{BCI} implemented as a practical part of this work. One possible usage of a \acrshort{BCI} that efficiently implements a communication channel between the brain and an external device would be to help severely disabled people to control devices that currently require pushing buttons, for example an electric wheelchair. The \acrshort{BCI} implemented as a part of this thesis uses widely known \acrshort{PSDA} and \acrshort{CCA} feature extraction methods and introduces a new way to combine these methods. Combining different methods improves the performance of a \acrshort{BCI}. The application was tested only superficially and the following results were obtained: $2.61\pm 0.28$ s target detection time, $85.81\pm 6.39$ \% accuracy and $27.73\pm 5.11$ bits/min \acrshort{ITR}. The implemented \acrshort{BCI} is open-source, written in Python 2.7, has graphical user interface and uses inexpensive \acrshort{EEG} device called Emotiv EPOC. The \acrshort{BCI} requires only a computer and Emotiv EPOC, no additional hardware is needed. Different \acrshort{EEG} devices could be used after modifying the code. 

\section*{Keywords}

\acrfull{EEG}, \acrfull{BCI}, \acrfull{SSVEP}, \acrfull{CCA}, \acrfull{PSDA}, open-source
\selectlanguage{estonian}
\chapter*{Visuaalse stiimuliga esilekutsutud potentsiaalidel põhinev roboti juhtimine Emotiv EPOC seadmega}

\section*{Lühikokkuvõte}

Antud töö kirjeldab visuaalse stiimuliga esilekutsutud potentsiaalidel põhinevat aju ning arvuti vahelist liidest (AAL), mis loodi antud töö praktilise osana. AALi saab kasutada aju ja seadme vahelise otsese suhtluskanali loomiseks, mis tähendab, et seadmega suhtlemiseks pole vaja nuppe vajutada, piisab vaid visuaalsete stiimulite vaatamisest. Efektiivne AAL võimaldaks raske puudega isikutel näiteks elektroonilist ratastooli juhtida. Antud töö osana loodud AAL kasutab tuntud kanoonilise korrelatsiooni- ja võimsusspektri analüüsi meetodeid ning uuendusena kombineerib need kaks meetodit üheks teineteist täiendavaks meetodiks. Kahe meetodi kombinatsioon muudab AALi täpsemaks. AALi testiti antud töös vaid pealiskaudselt ning tulemused on järgnevad: ühe käsu edastamise aeg $2.61\pm 0.28$ s, täpsus $85.81\pm 6.39$ \% ning informatsiooni edastamise kiirus $27.73\pm 5.11$ bits/min. Antud AAL on avatud lähtekoodiga, kirjutatud Python 2.7 programmeerimiskeeles, sisaldab graafilist kasutajaliidest ning kasutab aju tegevuse mõõtmiseks elektroensefalograafia (EEG) seadet Emotiv EPOC. AALi kasutamiseks on vaja ainult arvutit ja Emotiv EPOC seadet. Koodi muutes on võimalik kasutada ka teisi EEG seadmeid. 

\section*{Võtmesõnad}

Elektroensefalograafia \acrshort{EEG}, aju-arvuti liides (AAL), visuaalne stiimul, kanooniline korrelatisoonianalüüs, võimsusspektri analüüs, avatud lähtekood
\selectlanguage{english}
