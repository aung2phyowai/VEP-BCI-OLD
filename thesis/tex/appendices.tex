% Appendices
% Can be replaced by \appendix

\begin{appendices}
\setcounter{table}{0}
\renewcommand{\thetable}{A\arabic{table}}
%\titleformat{\chapter}{\large\bfseries}{\appendixname~\thesection .}{0.5em}{}
%\titleformat{\chapter}{\normalfont\LARGE\bfseries}{\appendixname~\thechapter}{20pt}{}{}
\renewcommand{\thesection}{\Roman{section}}
\section{Glossary}
\printglossary
\newpage
\section{Acronyms}
\printglossary[type=\acronymtype]
\newpage
\begin{landscape}
\section{Fixed parameters for testing}
\label{sec:parameters}
	\begin{table}[h]
		\centering
		\begin{tabular}{|l|l|l|l|l|l|l|l|l|l|l|}\hline
Method& Extraction& Length& Step	& Sensors		& Detrend	& Detrend	 & Window& Kaiser& Interpolation& Filter\\
ID	& method	& (sec)	& (sec)	& 				& 			& breakpoints& 		 & beta	 & 				&  \\\hline
1	& Sum PSDA	& 2		& 0.125	& O1, O2		& Linear	& 5			 & Kaiser& 14	 & Quadratic	&- \\\hline
2	& CCA		& 2		& 0.125	& O1, O2, P7, P8& Linear	& 5			 & -	 & -	 & -			&- \\\hline
3	& PSDA		& 1		& 0.125	& O1, O2		& Linear	& 3			 & Kaiser& 14	 & Quadratic	&- \\\hline
4	& CCA		& 1		& 0.125	& O1, O2, P7, P8& Linear	& 3			 & -	 & -	 & -			&- \\\hline
		\end{tabular}
		\caption{Settings of the feature extraction method's signal pipelines.}
		\label{tab:pipelines}
	\end{table}
	\begin{table}[h]
		\centering
		\begin{tabular}{|l|l|l|l|l|l|l|l|l|l|}\hline
Target	& Frequency & Harmonics		& Position x& Position y& Color	& Type			& Shape	 & Width 	& Height\\
ID		& (Hz)		& in extraction	& (pixel)	& (pixel)	& 		& 		 		& 		 & (pixel)	&(pixel)\\\hline
1		& 5.45		& 1, 2, 3		& 0		 	& -200		& White	& Single graphic& Square & 150	 	& 150\\\hline
2		& 6.0		& 1, 2, 3		& -300	 	& 0			& White	& Single graphic& Square & 150	 	& 150\\\hline
3		& 6.67		& 1, 2, 3		& 0		 	& 200		& White	& Single graphic& Square & 150	 	& 150\\\hline
4		& 7.5		& 1, 2, 3		& 300	 	& 0			& White	& Single graphic& Square & 150	 	& 150\\\hline
		\end{tabular}
		\caption{Settings of the targets.}
		\label{tab:targets}
	\end{table}
\end{landscape}
\newpage
\section{Code of the application}
\label{sec:code}

The application written as a practical part of this thesis is open-source and the code is accessible from Github repository\footnote{https://github.com/kahvel/VEP-BCI}. The detailed description of the application is in chapter~\ref{sec:SSVEP_BCI} of this thesis. For the most up to date instructions on which libraries are required, how the repository is structured and how to start using the application, see \texttt{README.md} file in the repository.
%\section{List of figures}
%\makeatletter
%\@starttoc{lof}% Print List of Figures
%\makeatother
%\newpage
%\listoftables

\end{appendices}
