
% Acronyms

\newacronym{EEG}{EEG}{electroencephalography}
\newacronym{BCI}{BCI}{brain-computer interface}
\newacronym{FFT}{FFT}{fast Fourier transform}
\newacronym{CCA}{CCA}{canonical correlation analysis}
\newacronym{MEG}{MEG}{magnetoencephalography}
\newacronym{fMRI}{fMRI}{functional magnetic resonance imaging}
\newacronym{PET}{PET}{positron emission tomography}
\newacronym{fNIRS}{fNIRS}{functional near-infrared spectroscopy}
\newacronym{ERP}{ERP}{event-related potential}
\newacronym{VEP}{VEP}{visual evoked potential}
\newacronym{SSVEP}{SSVEP}{steady-state visual evoked potential}
\newacronym{ADC}{ADC}{analog-to-digital converter}
\newacronym{SC}{SC}{stability coefficient}
\newacronym{PSDA}{PSDA}{power spectrum density analysis}
\newacronym{MEC}{MEC}{minimum energy combination}
\newacronym{LASSO}{LASSO}{least absolute shrinkage and selection operator}
\newacronym{LRT}{LRT}{likelihood ratio test}
\newacronym{MPCC}{MPCC}{multi-phase cycle coding}
\newacronym{LED}{LED}{light-emitting diodes}
\newacronym{SNR}{SNR}{signal-to-noise ratio}

% Glossary

\newglossaryentry{neuron}{
	name = neuron,
	description = {-- nerve cell}
}
\newglossaryentry{dendrite}{
	name = dendrite,
	description = {-- nerve ending}
}
\newglossaryentry{axon}{
	name = axon,
	description = {-- nerve fibre}
}
\newglossaryentry{synapse}{
	name = synapse,
	description = {-- connection between neurons}
}
\newglossaryentry{neural pathway}{
	name = neural pathway,
	description = {-- network formed by functionally related neurons}
}
\newglossaryentry{membrane potential}{
	name = membrane potential,
	description = {-- difference between the interior and the exterior of a cell}
}
\newglossaryentry{resting state}{
	name = resting state,
	description = {-- the state of a neuron that is not exited and not sending signals at the moment}
}
\newglossaryentry{resting potential}{
	name = resting potential,
	description = {-- the membrane potential of a neuron at a resting state}
}
\newglossaryentry{action potential}{
	name = action potential,
	description = {-- the event of sending signals used by neurons}
}
\newglossaryentry{ionophoric protein}{
	name = ionophoric protein,
	description = {-- proteins used to transport ions across cell membrane}
}
\newglossaryentry{threshold potential}{
	name = threshold potential,
	description = {-- certain threshold value. If membrane potential exceeds threshold potential, action potential is initiated}
}
\newglossaryentry{postsynaptic cell}{
	name = postsynaptic cell,
	description = {-- a cell that received a signal through synapse. This term is used for example when discussing single action potential and its effect on the cell that receives the signal}
}
\newglossaryentry{postsynaptic potential}{
	name = postsynaptic potential,
	description = {-- change in the membrane potential in a postsynaptic cell caused by synaptic input}
}
\newglossaryentry{sink}{
	name = sink,
	description = {-- location where ions enter the postsynaptic cell after synaptic input has been received}
}
\newglossaryentry{source}{
	name = source,
	description = {-- location where ions exit the postsynaptic cell after synaptic input has been received to achieve electroneutrality}
}
\newglossaryentry{current dipole}{
	name = current dipole,
	description = {-- formed by a sink and a source around postsynaptic cell. Dipoles produce electric field that can be measured from the scalp using EEG}
}
\newglossaryentry{functional neuroimaging}{
	name = functional neuroimaging,
	description = {-- measuring an aspect of brain function}
}
\newglossaryentry{haemodynamic}{
	name = haemodynamic,
	description = {--}
}
\newglossaryentry{electromagnetic}{
	name = electromagnetic,
	description = {--}
}
\newglossaryentry{temporal resolution}{
	name = temporal resolution,
	description = {--}
}
\newglossaryentry{visual processing centre}{
	name = visual processing centre,
	description = {--}
}
\newglossaryentry{primary visual pathway}{
	name = primary visual pathway,
	description = {--}
}
\newglossaryentry{central visual field}{
	name = central visual field,
	description = {--}
}
\newglossaryentry{peripheral vision}{
	name = peripheral vision,
	description = {--}
}
\newglossaryentry{fundamental frequency}{
	name = fundamental frequency,
	description = {--}
}
\newglossaryentry{sampling rate}{
	name = sampling rate,
	description = {--}
}
\newglossaryentry{flickering}{
	name = flickering,
	description = {--}
}
\newglossaryentry{target}{
	name = target,
	description = {--}
}
\newglossaryentry{feature extraction}{
	name = feature extraction,
	description = {--}
}
\newglossaryentry{square wave}{
	name = square wave,
	description = {--}
}
\newglossaryentry{rectangular wave}{
	name = rectangular wave,
	description = {--}
}
\newglossaryentry{duty cycle}{
	name = duty cycle,
	description = {--}
}
\newglossaryentry{harmonic}{
	name = harmonic,
	description = {--}
}
\newglossaryentry{subharmonic}{
	name = subharmonic,
	description = {--}
}
\newglossaryentry{fundamental}{
	name = fundamental,
	description = {--}
}
\newglossaryentry{refresh rate}{
	name = refresh rate,
	description = {--}
}
\newglossaryentry{frame}{
	name = frame,
	description = {--}
}
\newglossaryentry{flickering waveform}{
	name = flickering waveform,
	description = {--}
}
\newglossaryentry{state}{
	name = state,
	description = {--}
}
\newglossaryentry{single graphic}{
	name = single graphic,
	description = {--}
}
\newglossaryentry{pattern reversal}{
	name = single graphic,
	description = {--}
}
\newglossaryentry{pure tone}{
	name = pure tone,
	description = {--}
}
\newglossaryentry{Fourier transform}{
	name = Fourier transform,
	description = {--}
}
\newglossaryentry{frequency spectrum}{
	name = frequency spectrum,
	description = {--}
}
\newglossaryentry{power spectral density}{
	name = power spectral density,
	description = {--}
}
\newglossaryentry{periodogram}{
	name = periodogram,
	description = {--}
}
\newglossaryentry{Nyguist frequency}{
	name = Nyguist frequency,
	description = {--}
}
\newglossaryentry{frequency bin}{
	name = frequency bin,
	description = {--}
}
\newglossaryentry{frequency component}{
	name = frequency component,
	description = {--}
}
\newglossaryentry{interpolation}{
	name = interpolation,
	description = {--}
}
\newglossaryentry{zero padding}{
	name = zero padding,
	description = {--}
}