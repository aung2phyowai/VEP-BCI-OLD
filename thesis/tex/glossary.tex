
% Acronyms

\newacronym{BCI}{BCI}{brain-computer interface}
\newacronym{EEG}{EEG}{electroencephalography}
\newacronym{FFT}{FFT}{fast Fourier transform}
\newacronym{CCA}{CCA}{canonical correlation analysis}
\newacronym{MEG}{MEG}{magnetoencephalography}
\newacronym{fMRI}{fMRI}{functional magnetic resonance imaging}
\newacronym{PET}{PET}{positron emission tomography}
\newacronym{fNIRS}{fNIRS}{functional near-infrared spectroscopy}
\newacronym{ERP}{ERP}{event-related potential}
\newacronym{VEP}{VEP}{visual evoked potential}
\newacronym{SSVEP}{SSVEP}{steady-state visual evoked potential}
\newacronym{ADC}{ADC}{analog-to-digital converter}
\newacronym{SC}{SC}{stability coefficient}
\newacronym{PSDA}{PSDA}{power spectrum density analysis}
\newacronym{MEC}{MEC}{minimum energy combination}
\newacronym{LASSO}{LASSO}{least absolute shrinkage and selection operator}
\newacronym{LRT}{LRT}{likelihood ratio test}
\newacronym{MPCC}{MPCC}{multi-phase cycle coding}
\newacronym{LED}{LED}{light-emitting diode}
\newacronym{SNR}{SNR}{signal-to-noise ratio}
\newacronym{ITR}{ITR}{information transfer rate}
\newacronym{CPU}{CPU}{central processing unit}
\newacronym{GUI}{GUI}{graphical user interface}

% Glossary

\newglossaryentry{neuron}{
	name = neuron,
	description = {-- nerve cell, main building block of the brain.}
}
\newglossaryentry{dendrite}{
	name = dendrite,
	description = {-- nerve ending through which a \gls{neuron} receives signals.}
}
\newglossaryentry{axon}{
	name = axon,
	description = {-- nerve fibre through which a \gls{neuron} sends signals.}
}
\newglossaryentry{synapse}{
	name = synapse,
	description = {-- a connection between \glspl{neuron}.}
}
\newglossaryentry{neural pathway}{
	name = neural pathway,
	description = {-- network formed by functionally related \glspl{neuron}.}
}
\newglossaryentry{membrane potential}{
	name = membrane potential,
	description = {-- difference in electric potential between the inside of a \gls{neuron} and the extracellular fluid around the \gls{neuron}.}
}
\newglossaryentry{resting state}{
	name = resting state,
	description = {-- state of a \gls{neuron} that is not exited and not sending signals at the moment.}
}
\newglossaryentry{resting potential}{
	name = resting potential,
	description = {-- \gls{membrane potential} of a neuron at a resting state.}
}
\newglossaryentry{action potential}{
	name = action potential,
	description = {-- event of sending signals used by \glspl{neuron}.}
}
\newglossaryentry{ionophoric protein}{
	name = ionophoric protein,
	description = {-- proteins used to transport ions across cell membrane.}
}
\newglossaryentry{threshold potential}{
	name = threshold potential,
	description = {-- certain threshold value. If the \gls{membrane potential} exceeds threshold potential, \gls{action potential} is initiated.}
}
\newglossaryentry{postsynaptic cell}{
	name = postsynaptic cell,
	description = {-- a cell that received a signal through a \gls{synapse}. This term is used for example when discussing single \gls{action potential} and its effect on the cell that receives the signal.}
}
\newglossaryentry{postsynaptic potential}{
	name = postsynaptic potential,
	description = {-- change in the \gls{membrane potential} in a \gls{postsynaptic cell} caused by synaptic input.}
}
\newglossaryentry{sink}{
	name = sink,
	description = {-- part of the \gls{current dipole} that is more negatively charged than \gls{source}.}
}
\newglossaryentry{source}{
	name = source,
	description = {-- part of the \gls{current dipole} that is more positively charged than \gls{sink}.}
}
\newglossaryentry{current dipole}{
	name = current dipole,
	description = {-- formed by a \gls{sink} and a source around \gls{postsynaptic cell}. Dipoles produce electric field that can be measured from the scalp using \gls{EEG}.}
}
\newglossaryentry{functional neuroimaging}{
	name = functional neuroimaging,
	description = {-- measuring an aspect of brain function.}
}
\newglossaryentry{haemodynamic}{
	name = haemodynamic,
	description = {-- \gls{functional neuroimaging} technique that measures blood oxygenation and blood flow.}
}
\newglossaryentry{electromagnetic}{
	name = electromagnetic,
	description = {-- \gls{functional neuroimaging} technique that measures electric or magnetic fields.}
}
\newglossaryentry{temporal resolution}{
	name = temporal resolution,
	description = {-- the shortest time period of neural activity that is reliably separated out by a measuring technique.}
}
\newglossaryentry{visual processing centre}{
	name = visual processing centre,
	description = {-- part of the brain that is responsible for visual perception.}
}
\newglossaryentry{primary visual pathway}{
	name = primary visual pathway,
	description = {-- \gls{neural pathway} that is used to convey information that is essential for seeing.}
}
\newglossaryentry{central visual field}{
	name = central visual field,
	description = {-- very centre of gaze that is clearly seen, not like peripheral vision that is blurry.}
}
\newglossaryentry{peripheral vision}{
	name = peripheral vision,
	description = {-- not the centre of gaze but the area that surrounds the centre and is not as clearly seen.}
}
\newglossaryentry{fundamental frequency}{
	name = fundamental frequency,
	description = {-- frequency of the component of a signal that has the lowest frequency among all the frequency components.}
}
\newglossaryentry{sampling rate}{
	name = sampling rate,
	description = {-- rate at which samples are obtained for example from an \gls{EEG} device.}
}
\newglossaryentry{flickering}{
	name = flickering,
	description = {-- the blinking or the \gls{state} switches of a \gls{target}.}
}
\newglossaryentry{target}{
	name = target,
	description = {-- visual stimuli of a \gls{VEP}-based \gls{BCI}}
}
\newglossaryentry{feature extraction}{
	name = feature extraction,
	description = {-- analysing the \gls{EEG} recording in a \gls{VEP}-based \gls{BCI}. Feature extraction possibly gives multiple results and later the final result has to be chosen from these results.}
}
\newglossaryentry{square wave}{
	name = square wave,
	description = {-- periodic waveform that has \gls{duty cycle} of 50\%, is symmetric and the transition between maximum and minimum is instantaneous.}
}
\newglossaryentry{rectangular wave}{
	name = rectangular wave,
	description = {-- periodic waveform that is similar to the \gls{square wave}, but does not have the symmetrical shape, hence its \gls{duty cycle} is not 50\%. The transition between maximum and minimum is instantaneous.}
}
\newglossaryentry{duty cycle}{
	name = duty cycle,
	description = {-- the percentage of the amount of time the \gls{target} is in displayed \gls{state} in one period of the waveform.}
}
\newglossaryentry{harmonic}{
	name = harmonic,
	description = {-- \gls{frequency component} of a signal that has a frequency which is integer multiple of the \gls{fundamental} frequency.}
}
\newglossaryentry{subharmonic}{
	name = subharmonic,
	description = {-- \gls{frequency component} of a signal that has a frequency of which the \gls{fundamental} frequency is a integer multiple of.}
}
\newglossaryentry{fundamental}{
	name = fundamental,
	description = {-- frequency component of a signal that has the lowest frequency among all the frequency components.}
}
\newglossaryentry{refresh rate}{
	name = refresh rate,
	description = {-- the number of consecutive images shown on screen in one second.}
}
\newglossaryentry{frame}{
	name = frame,
	description = {-- the consecutive images shown on screen that make up the moving picture.}
}
\newglossaryentry{flickering waveform}{
	name = flickering waveform,
	description = {-- waveform obtained when plotting the \gls{state} switches of a \gls{target}.}
}
\newglossaryentry{state}{
	name = state,
	description = {-- state of a \gls{target}.}
}
\newglossaryentry{single graphic}{
	name = single graphic,
	description = {-- type of a \gls{target}. Single graphic \glspl{target} blink on the screen.}
}
\newglossaryentry{pattern reversal}{
	name = pattern reversal,
	description = {-- type of a \gls{target}. Pattern reversal \glspl{target} reverse their patterns on the screen.}
}
\newglossaryentry{pure tone}{
	name = pure tone,
	description = {-- waveform that contains only one frequency.}
}
\newglossaryentry{Fourier transform}{
	name = Fourier transform,
	description = {-- transforming a signal from time domain to frequency domain.}
}
\newglossaryentry{frequency spectrum}{
	name = frequency spectrum,
	description = {-- a function of amplitude and phase versus frequency of a signal.}
}
\newglossaryentry{power spectral density}{
	name = power spectral density,
	description = {-- a function of power versus frequency of a signal.}
}
\newglossaryentry{periodogram}{
	name = periodogram,
	description = {-- estimate of \gls{power spectral density}.}
}
\newglossaryentry{Nyguist frequency}{
	name = Nyguist frequency,
	description = {-- half the \gls{sampling rate}.}
}
\newglossaryentry{frequency bin}{
	name = frequency bin,
	description = {-- a value at which the \gls{frequency spectrum} or the \gls{power spectral density} function is defined.}
}
\newglossaryentry{frequency component}{
	name = frequency component,
	description = {-- a \gls{pure tone} that a signal contains.}
}
\newglossaryentry{interpolation}{
	name = interpolation,
	description = {-- approximating unknown values.}
}
\newglossaryentry{zero padding}{
	name = zero padding,
	description = {-- adding zeros to the end of the signal for example to get desired \glspl{frequency bin} in \gls{frequency spectrum}.}
}
\newglossaryentry{digital signal}{
	name = digital signal,
	description = {-- a sequence of values that were obtained from measuring the value of analogue signal in a certain time interval.}
}
\newglossaryentry{trend}{
	name = trend,
	description = {-- steady increase or decrease of values in a signal.}
}
\newglossaryentry{detrend}{
	name = detrend,
	description = {-- removing the \gls{trend} from a signal.}
}
\newglossaryentry{mean}{
	name = mean,
	description = {-- the measure of central tendency.}
}
\newglossaryentry{window}{
	name = window,
	description = {-- a function that is used to smooth the ends of a signal.}
}
\newglossaryentry{spectral leakage}{
	name = spectral leakage,
	description = {-- if a signal is divided into segments so that a component of the signal does not have an integer multiple of its period in a segment and \gls{FFT} is performed on this segment, then some of the power of that component's frequency will be distributed over other \glspl{frequency bin}.}
}
\newglossaryentry{statistic}{
	name = statistic,
	description = {-- a measure of an attribute of a \gls{sample}.}
}
\newglossaryentry{covariance}{
	name = covariance,
	description = {-- measure of similarity.}
}
\newglossaryentry{sample}{
	name = sample,
	description = {-- a set of data.}
}
\newglossaryentry{correlation}{
	name = correlation,
	description = {-- the Pearson product-moment correlation coefficient. Covariance divided by the product of standard deviations.}
}
\newglossaryentry{cross-correlation}{
	name = cross-correlation,
	description = {-- measure of similarity for two signals as a function of similarity versus time lag between the signals.}
}
\newglossaryentry{template matching}{
	name = template matching,
	description = {-- \gls{feature extraction} method which compares current signal to previously obtained templates.}
}
\newglossaryentry{reference signal}{
	name = reference signal,
	description = {-- a signal used in \gls{CCA} method to find certain frequencies in recorded signal by comparing the recorded signal to it.}
}
\newglossaryentry{canonical correlation}{
	name = canonical correlation,
	description = {-- correlation between the \glspl{linear combination} of two data samples.}
}
\newglossaryentry{linear combination}{
	name = linear combination,
	description = {-- for a sequence of values, each value is multiplied by a constant and the result is added up.}
}
