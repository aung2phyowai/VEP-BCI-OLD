
\chapter{Application for controlling a robot via SSVEP}
\label{sec:SSVEP_BCI}

The aim of this chapter is to describe the \gls{SSVEP}-based \gls{BCI} designed \dots. The \gls{BCI} is written in python 2.7 and the code is accessible from Github repository. See appendix~\ref{sec:bci_manual} for installation and usage instructions. The visual stimuli are designed using PsychoPy~\cite{psychopy}. \gls{CCA} algorithm is calculated using Scikit-Learn module~\cite{scikit-learn}. Most of other mathematical algorithms including \gls{FFT} are calculated using Scipy and Numpy modules~\cite{scipy}. The \gls{BCI} requires only Emotiv EPOC headset and a laptop computer, no specific hardware like digital signal processors or \glspl{LED} are used.

\section{Related work}

There are many types of \glspl{BCI} available. A review by Bashashati \textit{et al.}~\cite{bci_comparison} contains a detailed overview of \gls{EEG}-based \glspl{BCI}. Their paper includes overview of the following neuromechanisms used in \gls{EEG}-based \glspl{BCI}: \gls{VEP} and \gls{SSVEP}, P300, slow cortical potential, response to mental task, sensorimotor activity, and multiple neuromechanisms or hybrid \gls{BCI}. Hybrid \gls{BCI} uses two different neuromechanisms in a \gls{BCI} and therefore two different methods are required to analyse \gls{EEG} recording~\cite{hybrid_bci, hybrid_bci2}. %In this thesis, main focus is on \gls{SSVEP}-based \glspl{BCI}. 

\gls{SSVEP}-based \glspl{BCI} can be divided in categories according to the method used to detect \glspl{SSVEP} in \gls{EEG} recording. Current \gls{SSVEP}-based \gls{BCI} \gls{feature extraction} methods include:
\begin{itemize}
	\item \Gls{PSDA} method introduced by Cheng \textit{et al.}~\cite{psda}.
	\item \Gls{SC} method introduced by Wu and Yao~\cite{sc}.
	\item Dual-frequency \gls{SSVEP} methods~\cite{dual1, dual2}.
	\item \Gls{MPCC} method introduced by Tong \textit{et al.}~\cite{MPCC}.
	\item \Gls{MEC} method introduced by Friman \textit{et al.}~\cite{mec}. This method has shown better performance than \gls{SC} and \gls{PSDA} method~\cite{mec_comparison}.
	\item \Gls{CCA} method introduced by Lin \textit{et al.}~\cite{cca_lin}. This method has shown better performance than \gls{PSDA} method~\cite{cca_psda, bin2009cca, cca_lin}.
	\item \Gls{LASSO} method introduced by Zhang \textit{et al.}~\cite{LASSO}. This method has shown better performance than \gls{CCA} method~\cite{LASSO}.
	\item \Gls{LRT} method introduced by Zhang \textit{et al.}~\cite{LRT}. This method has shown better performance than \gls{CCA} method and similar performance to \gls{LASSO} method~\cite{LRT}.
\end{itemize}
There is also an improvement of \gls{CCA} method available called multiway \gls{CCA}, which has shown slightly better performance than standard \gls{CCA} method~\cite{mcca}. For more comprehensive review see article by Liu \textit{et al.}~\cite{feature_extraction}.

In this thesis, Emotiv EPOC is used to record brain activity. The following papers also describe \glspl{BCI} that used Emotiv EPOC to record brain activity:
\begin{itemize}
	\item Liu \textit{et al.}~\cite{emotiv_11hz} used \gls{CCA} method;
	\item Lin \textit{et al.}~\cite{emotiv_walking} used \gls{CCA} method;
	\item Choi and Jo~\cite{emotiv_hybrid} designed hybrid \gls{BCI};
	\item Zier~\cite{emotiv_psda} used \gls{PSDA} method;
	\item Hvaring and Ulltveit-Moe~\cite{emotiv_comparison} compared different \gls{feature extraction} methods;
	\item Duvinage \textit{et al.}~\cite{emotiv_p300_comp} used P300 method and compared the performance of Emotiv EPOC and a medical-grade device.
\end{itemize}
%This thesis focuses on two of these methods: \gls{PSDA} and \gls{CCA} methods.

\section{Signal pipeline}
\label{sec:signal_pipeline}

\subsubsection{Discrete spectrum}

One problem that occurs when using \gls{PSDA} is that the \gls{power spectral density} is discrete and therefore it might not have frequency bins at the values that correspond to \glspl{target} frequencies or its \glspl{harmonic}. In this case \gls{interpolation} can be used to approximate the value between frequency bins that exactly corresponds to \gls{target} frequency~\cite{cca_psda}. \Gls{interpolation} is used to construct a discrete signal between the discrete points or in other words to approximate the continuous signal from which the discrete values were extracted from.

Another possibility to solve this problem is to use \gls{zero padding}. \Gls{zero padding} means that zeros are added to the end of the discrete signal before performing \gls{FFT}. This results in more \glspl{frequency bin} in the \gls{power spectral density}, because the number of \glspl{frequency bin} depends on the length of the input signal. The only alteration in \gls{power spectral density} is that it has more \glspl{frequency bin} if calculated from zero-padded signal.

\subsubsection{Trend in the signal}

Linear trend or constant increase or decrease of values in the \gls{EEG} recording can make detecting the \gls{SSVEP} less accurate. Since Fourier analysis is used to decompose the recorded signal into \glspl{pure tone}, the linear trend present in the recording will also be decomposed. The decomposed trend will not provide any useful information and therefore the signal should be detrended before performing \gls{FFT}.

\subsubsection{Windowing the signal}

When estimating the \gls{power spectral density} of a signal using \gls{FFT}, it has to be decided how long signal segments are used or in other words, how many samples have to be acquired before performing \gls{FFT}. Dividing the signal into segments can make the \gls{SSVEP} detection less accurate, because some periodic components might

% phase in CCA
The phase information from the frequency spectrum can be used to make correct reference signals in \gls{CCA} method. Then it would not be necessary to use both sine and cosine waves and the correlation would be more accurate.

\section{Target identification}
asd
\section{The robot}
asd