
\chapter{SSVEP-based BCI for controlling a robot}
\label{sec:SSVEP_BCI}

The aim of this chapter is to describe the \gls{SSVEP}-based \gls{BCI} designed \dots. The \gls{BCI} is written in python 2.7 and the code is accessible from Github repository. See appendix~\ref{sec:bci_manual} for installation and usage instructions. The visual stimuli are designed using PsychoPy~\cite{psychopy}. \gls{CCA} algorithm is calculated using Scikit-Learn module~\cite{scikit-learn}. Most of other mathematical algorithms including \gls{FFT} are calculated using Scipy and Numpy modules~\cite{scipy}. The \gls{BCI} requires only Emotiv EPOC headset and a laptop computer, no specific hardware like digital signal processors or \glspl{LED} are used.

\section{Signal pipeline}
\label{sec:signal_pipeline}

\subsubsection{Discrete spectrum}

One problem that occurs when using \gls{PSDA} is that the \gls{power spectral density} is discrete and therefore it might not have frequency bins at the values that correspond to \glspl{target} frequencies or its \glspl{harmonic}. In this case \gls{interpolation} can be used to approximate the value between frequency bins that exactly corresponds to \gls{target} frequency~\cite{cca_psda}. \Gls{interpolation} is used to construct a discrete signal between the discrete points or in other words to approximate the continuous signal from which the discrete values were extracted from.

Another possibility to solve this problem is to use \gls{zero padding}. \Gls{zero padding} means that zeros are added to the end of the discrete signal before performing \gls{FFT}. This results in more \glspl{frequency bin} in the \gls{power spectral density}, because the number of \glspl{frequency bin} depends on the length of the input signal. The only alteration in \gls{power spectral density} is that it has more \glspl{frequency bin} if calculated from zero-padded signal.

\subsubsection{Trend in the signal}

Linear trend or constant increase or decrease of values in the \gls{EEG} recording can make detecting the \gls{SSVEP} less accurate. Since Fourier analysis is used to decompose the recorded signal into \glspl{pure tone}, the linear trend present in the recording will also be decomposed. The decomposed trend will not provide any useful information and therefore the signal should be detrended before performing \gls{FFT}.

\subsubsection{Windowing the signal}

When estimating the \gls{power spectral density} of a signal using \gls{FFT}, it has to be decided how long signal segments are used or in other words, how many samples have to be acquired before performing \gls{FFT}. Dividing the signal into segments can make the \gls{SSVEP} detection less accurate, because some periodic components might

\section{Target identification}
asd
\section{The robot}
asd