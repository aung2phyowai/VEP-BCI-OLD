
\chapter{Electroencephalography}
\section{Electricity in the brain}
As all known living organisms are composed of cells, so are humans and our brains. The brain consists of nerve cells, or neurons, and non-neural cells, called glial cells. There are approximately 86 billion neurons in the human brain and roughly as much non-neural cells \cite{neuroncount}.

There are connections between neurons, through which neurons can interact with each other. These connections are not static and can change over time. A typical neuron has a cell body, multiple nerve endings, or dendrites, and not more than one nerve fibre, or axon.

Both, dendrites and axons can branch multiple times, but axon is much longer. In humans axon can be up to 1 meter long. The general rule is that neuron sends signals through axon and receives signals through dendrites. The connection between axon and dendrite is called synapse. 

There are sodium and potassium cations inside and outside the neuron to maintain electrical potentials. The concentration of potassium cations is higher in the exterior of the cell while the concentration of sodium is higher in the interior of the cell. When neurons are not sending signals, they have slightly negative potential which is called resting potential. This means that the difference in electric potential between the interior and the exterior of the cell is negative. 

The negative membrane potential is achieved by having less cations inside the cell than outside. To keep the interior potential stable, a sodium-potassium pump is used. This pump transports three sodium cations out of the cell for every two potassium cations pumped in.

Neurons interact with each other by sending electro-chemical signals. The signals are sent by rapidly changing the membrane potential along the axon. This event is called an action potential and it is generated by voltage-gated ion channels.

These channels are similar to sodium-potassium pumps but differ in the direction of ion transportation. Former is a active transport mechanism, meaning it moves ions across the cell membrane against their concentration gradient, while latter is passive transport mechanism which moves ions down the concentration gradient.

Voltage-gated ion channels, as the name suggests, are activated by a certain threshold voltage. When the membrane potential is near resting potential the channels are shut. 

Neurons that are functionally related connect to each other and form neural networks.
A typical neuron fires 5 - 50 times every second.
\section{Recording brain activity}
\section{Visual evoked potentials}
 