
\chapter{Electrical activity in the brain}

In this chapter we are going to discuss ...

\section{Source of the electrical activity}
\label{sec:neuron}

As all known living organisms are composed of cells, so are humans and the human brain. The brain consists of nerve cells, or neurons, and non-neural cells, called glial cells. There are approximately 86 billion neurons in the human brain and roughly as much non-neural cells \cite{neuroncount}. A typical neuron has a cell body, multiple nerve endings, or dendrites, and one nerve fibre, or axon. Both, dendrites and axons can branch multiple times, but axons can be much longer than dendrites. 

There are connections between neurons, through which neurons can interact with each other by sending electro-chemical signals. These connections are not static and can change over time. Functionally related neurons are connected to each other and form neural pathways \cite{neuralpathway}. The general rule is that a neuron sends signals through its axon and receives signals through dendrites. The connection between an axon and a dendrite is called a synapse. See figure \ref{fig:synapse} for an example of a neuron and a synapse.

\begin{figure}[b!]
	\begin{minipage}{0.5\textwidth}
		\centering
		\includegraphics[width=\textwidth]{synapse.jpg}
		\caption{Neurons and chemical synapse \cite[p.~17]{neuronpic}}
		\label{fig:synapse}
	\end{minipage}
	\begin{minipage}{0.5\textwidth}
		\centering
		%\includegraphics[width=\textwidth]{ActionPotential.png}
		\caption{Resting and action potential}
		\label{fig:action_potential}
	\end{minipage}
\end{figure}

To send signals, neurons must be able to maintain electrical potential, called membrane potential. Membrane potential is defined as the difference in electric potential between the interior and exterior of a cell. When neurons are not sending signals, they have slightly negative membrane potential which is called resting potential. Negative resting potential is achieved by having more positively charged ions inside the cell than around it. The positively charged ions are sodium and potassium cations.

The concentration of potassium ions is higher in the interior of a cell, while the concentration of sodium ions is higher in the exterior. To maintain negative resting potential, the sodium-potassium pump is used. This pump transports three sodium cations out of the cell for every two potassium cations pumped in. Sodium-potassium pump is an active transport mechanism, which means that it moves ions across the cell membrane against their concentration gradient.

By having stable resting potential, neuron is able to send signals by rapidly increasing and decreasing the membrane potential along an axon. Therefore, the signal travelling along an axon is actually higher membrane potential. This event is called an action potential. To increase the membrane potential, voltage-gated ion channels are used. These channels are similar to sodium-potassium pumps, but differ in the direction of ion transportation. Ion channels are passive transport mechanisms, which means they move ions across the cell membrane down their concentration gradient.

Voltage-gated ion channels, as the name suggests, are activated by a certain threshold voltage, called threshold potential. When the membrane potential is near resting potential the channels are shut. Synaptic inputs, however, can cause the membrane potential to increase and if the membrane potential exceeds the threshold, first voltage-gated ion channels are opened and sodium ions start entering the cell. Sodium ions further increase the membrane potential which causes next ion channels to be opened along the axon and more sodium ions can enter the cell and so on.

As sodium ions enter the cell, potassium ion channels start to open and potassium ions start diffusing out of the cell. The depolarisation occurs, because more sodium ions enter the cell than potassium ions diffuse out of the cell. After depolarisation, sodium ion channels are closed while potassium ion channels remain open. This causes the repolarisation of the cell, meaning that the membrane potential of the cell returns to a negative value, which is slightly lower than the resting potential. After potassium ion channels close, active transport reestablishes the resting potential of the cell.

When the action potential reaches the synapse, increased membrane potential causes the voltage-gated calcium channels to open. This increases calcium concentration in the interior of the cell which causes calcium-sensitive proteins to change shape and as a result, neurotransmitters are dumped into the synaptic cleft. Some of the neurotransmitter gets bound on the receptors of the signal receiving cell. See figure \ref{fig:synapse} for graphical illustration.

The last step of the signal sending is the action of the receptor, which might be opening ion channels in the neuron that receives the signal. Opened channels cause changes in the membrane potential in the receiving cell and might lead to another action potential.

\section{Functional neuroimaging}
\label{sec:neuroimaging}

As discussed in section \ref{sec:neuron}, neurons in the brain are sending electrochemical signals to communicate with each other. There are several techniques available to measure this activity and before we continue to discuss the biological background we will be giving a brief overview of these methods. Measuring an aspect of brain function is called functional neuroimaging and common measurement methods divide into haemodynamic and electromagnetic techniques.

Haemodynamic techniques measure blood oxygenation and blood flow in the brain. More oxygen has to be delivered to more active brain regions and this allows the brain activity to be measured. Haemodynamic techniques include functional \acrfull{fMRI}, \acrfull{fNIRS}, \acrfull{PET}.

Electromagnetic techniques measure either electrical activity or magnetic fields produced by the electrical activity along the scalp. Electromagnetic techniques include \acrfull{EEG} and \acrfull{MEG}. These methods have lower temporal resolution than haemodynamic methods, but measure only the activity in the outer layer of the brain. Temporal resolution is the smallest time period of neural activity that is reliably separated out by measuring technique.
%http://psychcentral.com/lib/types-of-brain-imaging-techniques/0001057
%http://www.macalester.edu/academics/psychology/whathap/UBNRP/Imaging/eeg.html
%http://scienceforthemasses.org/2014/04/11/selecting-an-eeg-device/

To decide which method is best for controlling a robot, we will compare cost, portability and temporal resolution of each method. See table \ref{tab:neuroimaging} for details. For real time robot controlling, lower temporal resolution is better, because it enables faster decision making. We also prefer techniques that are portable and available for the consumer, so the usage would not be limited to certain location and would be available to the people in need. Considering all these arguments, it can be seen that the best choice for controlling a robot is an \acrshort{EEG} device.

% Constants for table

\newcommand{\pMEG}{\tablefootnote{http://neurogadget.com/2012/12/15/inexpensive-magnetoencephalography-meg-system-could-be-available-at-every-hospital/6495}}
\newcommand{\pfMRI}{\tablefootnote{http://info.blockimaging.com/bid/92623/MRI-Machine-Cost-and-Price-Guide}}
\newcommand{\pPET}{\tablefootnote{http://info.blockimaging.com/bid/68875/How-Much-Does-a-PET-CT-Scanner-Cost}}
\newcommand{\plEEG}{\tablefootnote{http://en.wikipedia.org/wiki/Comparison\_of\_consumer\_brain-computer\_interfaces}}
\newcommand{\phEEG}{\tablefootnote{http://www.brainvision.com/files/actiCHamp-PyCorder-Flyer\_US.pdf}}
\newcommand{\pNIRS}{\cite{NIRS}}
\newcommand{\tresol}{\cite{timeresol}}

% Table

\begin{table}[h]
	\centering
	\begin{tabular}{|c|c|c|c|c|c|}
	\hline
				& Price	from				& Portable	& Temporal resolution	& Special requirements		\\\hline
\acrshort{MEG}	& millions\pMEG				& no	& milliseconds \tresol		& magnetically shielded room\\\hline
\acrshort{fMRI}	& \SI{150000}[\$]\pfMRI		& no	& about 1 second \tresol	& magnetically shielded room\\\hline
\acrshort{PET}	& \SI{125000}[\$]\pPET		& no	& about 1 second \tresol	& radioactive isotopes injection\\\hline
\acrshort{fNIRS}& \SI{10000}[\$]{} \pNIRS	& yes	& hundreds of milliseconds \pNIRS&						\\\hline
\acrshort{EEG}	& \SI{100}[\$]\plEEG		& yes	& milliseconds \tresol		&							\\\hline
	\end{tabular}
	\caption{Comparison of functional neuroimaging methods}
	\label{tab:neuroimaging}
\end{table}

\section{Electroencephalography}

As already mentioned in the previous section, \acrshort{EEG} measures the electrical activity along the scalp. This electrical activity originates from the communication between neurons, as discussed in section \ref{sec:neuron}. 

\section{Visual evoked potential}

\section{Emotiv EPOC}
In section \ref{sec:neuroimaging} we briefly compared different functional neuroimaging methods and concluded that currently \acrshort{EEG} is most suitable for our needs. But there is a wide variety of EEG devices available. 
 