\documentclass[a4paper,12pt]{report}
\usepackage{myStyle}

% Acronyms

\newacronym{EEG}{EEG}{electroencephalography}
\newacronym{BCI}{BCI}{brain-computer interface}
\newacronym{FFT}{FFT}{fast Fourier transform}
\newacronym{CCA}{CCA}{canonical correlation analysis}
\newacronym{MEG}{MEG}{magnetoencephalography}
\newacronym{fMRI}{fMRI}{functional magnetic resonance imaging}
\newacronym{PET}{PET}{positron emission tomography}
\newacronym{fNIRS}{fNIRS}{functional near-infrared spectroscopy}
\newacronym{ERP}{ERP}{event-related potential}
\newacronym{VEP}{VEP}{visual evoked potential}
\newacronym{SSVEP}{SSVEP}{steady-state visual evoked potential}
\newacronym{ADC}{ADC}{analog-to-digital converter}
\newacronym{SC}{SC}{stability coefficient}
\newacronym{PSDA}{PSDA}{power spectrum density analysis}
\newacronym{MEC}{MEC}{minimum energy combination}
\newacronym{LASSO}{LASSO}{least absolute shrinkage and selection operator}
\newacronym{LRT}{LRT}{likelihood ratio test}
\newacronym{MPCC}{MPCC}{multi-phase cycle coding}
\newacronym{LED}{LED}{light-emitting diodes}
\newacronym{SNR}{SNR}{signal-to-noise ratio}

\newglossaryentry{neuron}{
	name = neuron,
	description = {-- nerve cell}
}
\newglossaryentry{dendrite}{
	name = dendrite,
	description = {-- nerve ending}
}
\newglossaryentry{axon}{
	name = axon,
	description = {-- nerve fibre}
}
\newglossaryentry{synapse}{
	name = synapse,
	description = {-- connection between neurons}
}
\newglossaryentry{neural pathway}{
	name = neural pathway,
	description = {-- network formed by functionally related neurons}
}
\newglossaryentry{membrane potential}{
	name = membrane potential,
	description = {-- difference between the interior and the exterior of a cell}
}
\newglossaryentry{resting state}{
	name = resting state,
	description = {-- the state of a neuron that is not exited and not sending signals at the moment}
}
\newglossaryentry{resting potential}{
	name = resting potential,
	description = {-- the membrane potential of a neuron at a resting state}
}
\newglossaryentry{action potential}{
	name = action potential,
	description = {-- the event of sending signals used by neurons}
}
\newglossaryentry{ionophoric protein}{
	name = ionophoric protein,
	description = {-- proteins used to transport ions across cell membrane}
}
\newglossaryentry{threshold potential}{
	name = threshold potential,
	description = {-- certain threshold value. If membrane potential exceeds threshold potential, action potential is initiated}
}
\newglossaryentry{postsynaptic cell}{
	name = postsynaptic cell,
	description = {-- a cell that received a signal through synapse. This term is used for example when discussing single action potential and its effect on the cell that receives the signal}
}
\newglossaryentry{postsynaptic potential}{
	name = postsynaptic potential,
	description = {-- change in the membrane potential in a postsynaptic cell caused by synaptic input}
}
\newglossaryentry{sink}{
	name = sink,
	description = {-- location where ions enter the postsynaptic cell after synaptic input has been received}
}
\newglossaryentry{source}{
	name = source,
	description = {-- location where ions exit the postsynaptic cell after synaptic input has been received to achieve electroneutrality}
}
\newglossaryentry{current dipole}{
	name = current dipole,
	description = {-- formed by a sink and a source around postsynaptic cell. Dipoles produce electric field that can be measured from the scalp using EEG}
}
\newglossaryentry{functional neuroimaging}{
	name = functional neuroimaging,
	description = {-- measuring an aspect of brain function}
}
\newglossaryentry{haemodynamic}{
	name = haemodynamic,
	description = {--}
}
\newglossaryentry{electromagnetic}{
	name = electromagnetic,
	description = {--}
}
\newglossaryentry{temporal resolution}{
	name = temporal resolution,
	description = {--}
}
\newglossaryentry{visual processing centre}{
	name = visual processing centre,
	description = {--}
}
\newglossaryentry{primary visual pathway}{
	name = primary visual pathway,
	description = {--}
}
\newglossaryentry{central visual field}{
	name = central visual field,
	description = {--}
}
\newglossaryentry{peripheral vision}{
	name = peripheral vision,
	description = {--}
}
\newglossaryentry{fundamental frequency}{
	name = fundamental frequency,
	description = {--}
}
\newglossaryentry{sampling rate}{
	name = sampling rate,
	description = {--}
}
\newglossaryentry{flickering}{
	name = flickering,
	description = {--}
}
\newglossaryentry{target}{
	name = target,
	description = {--}
}
\newglossaryentry{feature extraction}{
	name = feature extraction,
	description = {--}
}
\newglossaryentry{square wave}{
	name = square wave,
	description = {--}
}
\newglossaryentry{rectangular wave}{
	name = rectangular wave,
	description = {--}
}
\newglossaryentry{duty cycle}{
	name = duty cycle,
	description = {--}
}
\newglossaryentry{harmonic}{
	name = harmonic,
	description = {--}
}
\newglossaryentry{subharmonic}{
	name = subharmonic,
	description = {--}
}
\newglossaryentry{fundamental}{
	name = fundamental,
	description = {--}
}
\newglossaryentry{refresh rate}{
	name = refresh rate,
	description = {--}
}
\newglossaryentry{frame}{
	name = frame,
	description = {--}
}
\newglossaryentry{flickering waveform}{
	name = \gls{flickering} waveform,
	description = {--}
}
\newglossaryentry{state}{
	name = state,
	description = {--}
}
\newglossaryentry{single graphic}{
	name = single graphic,
	description = {--}
}
\newglossaryentry{pattern reversal}{
	name = single graphic,
	description = {--}
}
\newglossaryentry{pure tone}{
	name = pure tone,
	description = {--}
}
\newglossaryentry{Fourier transform}{
	name = Fourier transform,
	description = {--}
}
\newglossaryentry{frequency spectrum}{
	name = frequency spectrum,
	description = {--}
}
\newglossaryentry{power spectral density}{
	name = power spectral density,
	description = {--}
}
\newglossaryentry{periodogram}{
	name = periodogram,
	description = {--}
}
\newglossaryentry{Nyguist frequency}{
	name = Nyguist frequency,
	description = {--}
}
\newglossaryentry{frequency bin}{
	name = frequency bin,
	description = {--}
}
\newglossaryentry{frequency component}{
	name = frequency component,
	description = {--}
}
\newglossaryentry{interpolation}{
	name = interpolation,
	description = {--}
}
\newglossaryentry{zero padding}{
	name = zero padding,
	description = {--}
}

% Packages for layout testing

%\usepackage{showframe}
%\usepackage{layouts}
%\usepackage{layout}
%\printinunitsof{pt}

\includeonly{./tex/abstract, ./tex/title_page, ./tex/appendices, ./tex/introduction, ./tex/EEG, ./tex/BCI, ./tex/conclusion, ./tex/SSVEP_BCI, ./tex/results}
\begin{document}

%\layout
%\pagevalues
%\newpage

% Title page
% font size table: http://tex.stackexchange.com/questions/24599/what-point-pt-font-size-are-large-etc

\begin{center}
\thispagestyle{empty}
\large{UNIVERSITY OF TARTU}\\
\large{FACULTY OF MATHEMATICS AND COMPUTER SCIENCE}\\
\large{Institute of Computer Science}\\
\large{Computer Science Curriculum}\\
\vspace{160pt}
\Large{\bf Anti Ingel}\\
\vspace{6pt}
\LARGE{\textbf{Control a Robot via VEP Using \\Emotiv EPOC}}\\
\vspace{12pt}
\large{\bf Bachelor's Thesis (9 ECTS)}
\vspace{63pt}
\end{center}
\hfill \Large{Supervisor: Ilya Kuzovkin, MSc}
\vfill
\begin{center}
\large{Tartu 2015}
\end{center}


\chapter*{Control a robot via VEP using Emotiv EPOC}
\section*{Abstract}
\section*{Keywords}
\normalsize{EEG, BCI, f-VEP, SSVEP, CCA, PSDA, FFT, DFT, electroencephalography}
\section*{[Kommentaarid?]}
\section*{Lühikokkuvõte}
\section*{Võtmesõnad}
\tableofcontents
\addcontentsline{toc}{chapter}{Introduction}

\chapter*{Introduction}

Over the last decade the question how to implement direct communication channel between the brain and an external device has received much attention~\cite{mec, cca_lin, MPCC, sc, LRT, LASSO}. This communication channel works by translating recorded brain activity into commands to control the device or the other way round---by sending signals into the brain. Sending signals into the brain requires a surgery to implant the device that can send those signals. This approach can be used for example to repair damaged vision or hearing.

Measuring the brain activity, however, can be done without implanting anything into the brain and therefore it is easier and less expensive than invasive methods. This approach can be used for example to operate a wheelchair or for typing without having to press buttons. Nowadays the brain activity can be measured using non-invasive and inexpensive devices and thus many people could benefit from an application that efficiently implements this new communication channel between the brain and an external device.

This thesis describes an application that uses an \gls{EEG} device called Emotiv EPOC to measure brain activity and translates the recorded signal into commands to control a robot. The application was written as a practical part of this thesis and the focus was on implementing an inexpensive application. Compared to existing applications, the practical part of this thesis combines two widely used methods called \gls{PSDA}~\cite{psda} and \gls{CCA}~\cite{cca_lin} for extracting information from brain recording in a way which to the best of the author's knowledge has not been done before.

The first chapter of this thesis describes the biology of the brain and discusses, what aspect of the brain activity can actually be measured. In this chapter also different techniques and devices that can be used to measure brain activity are compared.

The second chapter describes how to evoke a brain response that can be extracted from the recording of brain activity using only Emotiv EPOC and a laptop computer. The methods that can be used to analyse the recording are discussed in this chapter.

The third chapter describes the application implemented as a practical part of this thesis. This chapter contains overview of related works, describes the signal processing and explains the novelty of the application.


\chapter{Electrical activity in the brain}

In this chapter we are going to discuss ...

\section{Source of the electrical activity}
\label{sec:neuron}

As all known living organisms are composed of cells, so are humans and the human brain. The brain consists of nerve cells and non-neural cells, called neurons and glial cells respectively. There are approximately 86 billion neurons in the human brain and roughly as much non-neural cells \cite{neuroncount}. A typical neuron has a cell body, multiple nerve endings, or dendrites, and one nerve fibre, or axon. Both, dendrites and axons can branch multiple times, but axons can be much longer than dendrites. 

There are connections between neurons, through which neurons can interact with each other by sending electro-chemical signals. These connections are not static and can change over time. Functionally related neurons are connected to each other and form neural pathways \cite{neuralpathway}. The general rule is that a neuron sends signals through its axon and receives signals through dendrites. The connection between an axon and a dendrite is called a synapse. See figure \ref{fig:neuron_synapse} for an example of a neuron and a synapse.

\begin{figure}[b!]
	\centering
	\begin{subfigure}{0.48\textwidth}
		\includegraphics[width=\textwidth]{synapse_modified.jpg}
		\caption{Neurons and a chemical synapse \cite[p.~17]{neuronpic}}
		\label{fig:neuron_synapse}
	\end{subfigure}
	~
	\begin{subfigure}{0.48\textwidth}
		\includegraphics[width=\textwidth]{action_potential.png}
		\caption{Action potential \cite{action_potential_pic}}
		\label{fig:action_potential}
	\end{subfigure}
	\caption{The structure of a neuron}
\end{figure}

To send signals, neurons must be able to maintain electrical potential, called membrane potential. Membrane potential is the difference in electric potential between the interior and exterior of a cell. When neurons are not sending signals, they have slightly negative membrane potential which is called resting potential. Negative resting potential is achieved by having more positively charged ions inside the cell than around it.

By having stable resting potential, neuron is able to send signals by rapidly increasing and decreasing the membrane potential along an axon. Therefore, the signal travelling along an axon is actually higher membrane potential. This event is called an action potential. To increase or decrease the membrane potential of a cell, ionophoric proteins are used. These proteins transport ions across the cell membrane.

Action potential is initiated by a certain threshold voltage, called threshold potential. When the membrane potential of a neuron exceeds threshold potential, the neuron sends signals to other cells. The neuron that receives the signal is called postsynaptic cell. Synaptic input in a postsynaptic cell can cause the membrane potential of the postsynaptic cell to increase or decrease. This is called postsynaptic potential and it can cause an action potential in the postsynaptic cell.

The following paragraph is based on article \cite{electric_field}. Postsynaptic potentials cause small current flows into the cell. To achieve electroneutrality, a balancing current from the interior to the exterior of the cell is needed. Location, where ions enter the cell is called sink and where ions flow to the exterior of the cell is called source. A source and a sink form a dipole. This is important, because it is possible to measure the electric field produced by these dipoles from the scalp. Although action potentials generate stronger currents, their duration is low and nearby neurons rarely fire synchronously.

\begin{figure}[b!]
	\centering
	\begin{subfigure}{0.48\textwidth}
		\includegraphics[width=\textwidth]{dipole_neuron.png}
		\caption{A dipole generated by a neuron \cite[p.~669]{neuroscience}}
		\label{fig:dipole_neuron}
	\end{subfigure}
	~
	\begin{subfigure}{0.48\textwidth}
		\includegraphics[width=\textwidth]{dipole_electric.png}
		\caption{The electric field of a dipole} %{http://hyperphysics.phy-astr.gsu.edu/hbase/electric/equipot.html}
		\label{fig:dipole_electric}
	\end{subfigure}
	\caption{Dipoles}
	\label{fig:dipole}
\end{figure}

\section{Functional neuroimaging}
\label{sec:neuroimaging}

As discussed in section \ref{sec:neuron}, neurons in the brain are sending electrochemical signals to communicate with each other. There are several techniques available to measure this activity and before we continue to discuss the biological background we will be giving a brief overview of these methods. Measuring an aspect of brain function is called functional neuroimaging and common measurement methods divide into haemodynamic and electromagnetic techniques.

Haemodynamic techniques measure blood oxygenation and blood flow in the brain. More oxygen has to be delivered to more active brain regions and this allows the brain activity to be measured. Haemodynamic techniques include \gls{fMRI}, \gls{fNIRS}, \gls{PET}.

Electromagnetic techniques measure either electrical activity or magnetic fields produced by the electrical activity along the scalp. Electromagnetic techniques include \gls{EEG} and \gls{MEG}. These methods have lower temporal resolution than haemodynamic methods, but measure only the activity in the outer layer of the brain. Temporal resolution is the smallest time period of neural activity that is reliably separated out by measuring technique.
%http://psychcentral.com/lib/types-of-brain-imaging-techniques/0001057
%http://www.macalester.edu/academics/psychology/whathap/UBNRP/Imaging/eeg.html
%http://scienceforthemasses.org/2014/04/11/selecting-an-eeg-device/

To decide which method is best for controlling a robot, we will compare cost, portability and temporal resolution of each method. See table \ref{tab:neuroimaging} for details. For real time robot controlling, lower temporal resolution is better, because it enables faster decision making. We also prefer techniques that are portable and available for the consumer, so the usage would not be limited to certain location and would be available to the people in need. Considering all these arguments, it can be seen that the best choice for controlling a robot is an \gls{EEG} device.

% Constants for table

\newcommand{\pMEG}{\tablefootnote{http://neurogadget.com/2012/12/15/inexpensive-magnetoencephalography-meg-system-could-be-available-at-every-hospital/6495}}
\newcommand{\pfMRI}{\tablefootnote{http://info.blockimaging.com/bid/92623/MRI-Machine-Cost-and-Price-Guide}}
\newcommand{\pPET}{\tablefootnote{http://info.blockimaging.com/bid/68875/How-Much-Does-a-PET-CT-Scanner-Cost}}
\newcommand{\pEEG}{\tablefootnote{http://en.wikipedia.org/wiki/Comparison\_of\_consumer\_brain-computer\_interfaces}}
\newcommand{\pNIRS}{\cite{NIRS}}
\newcommand{\tresol}{\cite{timeresol}}

% Table

\begin{table}[h]
	\centering
	\begin{tabular}{|c|c|c|c|c|c|}\hline
			& Price	from				& Portable	& Temporal resolution		& Special requirements			\\\hline
\gls{MEG}	& millions\pMEG				& no		& milliseconds \tresol		& magnetically shielded room	\\\hline
\gls{fMRI}	& \SI{150000}[\$]\pfMRI		& no		& about 1 second \tresol	& magnetically shielded room	\\\hline
\gls{PET}	& \SI{125000}[\$]\pPET		& no		& about 1 second \tresol	& radioactive isotopes injection\\\hline
\gls{fNIRS}	& \SI{10000}[\$]{} \pNIRS	& yes		& over 0.1 second \pNIRS	&								\\\hline
\gls{EEG}	& \SI{80}[\$]\pEEG			& yes		& milliseconds \tresol		&								\\\hline
	\end{tabular}
	\caption{Comparison of functional neuroimaging methods}
	\label{tab:neuroimaging}
\end{table}

\section{Electroencephalography}
\label{sec:EEG}

As already mentioned in the previous section, \gls{EEG} measures the electrical activity along the scalp. This electrical activity originates mainly from the electric fields generated by neurons, as discussed in section \ref{sec:neuron}. However, the electric potential generated by one neuron is far too low to be recognized. Therefore, approximately 108 neurons have to have synchronous electrical activity to a create measurable field \cite{field_count}. Furthermore, these neurons have to have certain orientation for the electric fields to add up and reach the electrode on the scalp. See figure \ref{fig:dipole_electric} for example.

\gls{EEG} measures the potential fields as the function of voltage versus time, using electrodes placed on scalp \cite{field_count}. Since voltage is the electrical potential difference between two points, one or more reference electrodes should be used. Then voltmeter can be used to measure the differences in voltage between two electrodes, one of which is a reference electrode.

Usually electrodes are placed on the scalp according to international 10-20 electrode location system. The outer layer of the brain can be classified into four lobes: temporal, occipital, parietal and frontal. 10-20 electrode location system uses a letter and a number to identify electrode location. The letter is the first letter of the brain lobe above which the electrode is located and therefore, the electrode measures the activity of this brain lobe.

In a broad sense, \gls{EEG} recording is linked to the state of the general activation of the brain \cite{VEP}. Therefore, due to the generality of the recording, we cannot see potentials evoked by certain event in the recording, because the general fluctuations are much larger than the evoked potential. A brain potential evoked by some event is called \gls{ERP}. \glspl{ERP} are associated with the information flow in the cortical areas and are usually obtained by an averaging technique \cite{ERP}. \glspl{ERP} are time locked to a stimulus event and therefore, may be recorded by presenting a stimulus with a certain time interval to a subject and calculating the average of \gls{EEG} signals recorded in the same time interval.

\begin{figure}[h]
	\centering
	\includegraphics[width=0.70\textwidth]{voltmeter.png}
	\caption{Voltmeter \cite[p.~120]{ERP}}
	\label{fig:voltmeter}
\end{figure}

\section{Visual evoked potential}

In section \ref{sec:EEG} we gave a brief overview of \glspl{ERP}. Now we are going to discuss one specific \gls{ERP}, called \gls{VEP}. As the name suggests, \glspl{VEP} are elicited by visual stimuli. The visual stimulus for eliciting a \gls{VEP} can be very simple, for example a white square blinking on a black computer screen.

The visual processing centre in humans is located in the back of the brain. The corresponding brain lobe is called occipital lobe. Therefore, when subject sees the stimulus, the signal travels from the eyes to the visual processing centre through the primary visual pathway \cite{neuroscience}. The primary visual pathway is a neural pathway. Neural pathways were also mentioned in the section \ref{sec:neuron}.

Since the occipital lobe is located in the back of the brain, when recording \glspl{VEP} with \gls{EEG}, electrodes located in the back of the head should be used. These electrode locations are identified with the letter O, as discussed in section \ref{sec:EEG}.

When comparing the \glspl{VEP} elicited in the central visual field and in the peripheral vision by the same stimuli, it can be seen that the stimulation of the central visual field produces larger \glspl{VEP} \cite{VEP_size}. This property is beneficial when presenting more than one stimulus to a subject and trying to identify, which stimulus is in the central visual field of the subject, or in other words, which stimulus is the subject looking at. 

To make the detection of \glspl{VEP} easier, \glspl{SSVEP} are used. If the visual stimulus is presented at a constant rate and the rate is so fast that the visual pathway cannot recover before the next stimulus is presented, then the elicited response becomes continuous and it is called the \gls{SSVEP} \cite{VEP}. See figure X for an example of \gls{VEP} and \gls{SSVEP}. \gls{SSVEP} is sinusoidal and therefore, it is easier to detect it in the \gls{EEG} recording.

\begin{figure}[h]
	\centering
	\begin{subfigure}{0.48\textwidth}
		\includegraphics[width=\textwidth]{brain_lobes.png}
		\caption{Lobes of the brain \cite{blausen}}
		\label{fig:brain_lobes}
	\end{subfigure}
	~
	\begin{subfigure}{0.48\textwidth}
		\includegraphics[width=\textwidth]{visual_pathway.png}
		\caption{The primary visual pathway \cite[p.~261]{neuroscience}}
		\label{fig:visual_pathway}
	\end{subfigure}
	\caption{The location of occipital lobe and the neural pathway from eyes to the lobe}
	\label{fig:lobes_pathway}
\end{figure}

\section{Emotiv EPOC}

In section \ref{sec:neuroimaging} we briefly compared different functional neuroimaging methods and concluded that currently \gls{EEG} is most suitable for our needs. But there is a wide variety of EEG devices available. See table \ref{tab:EEG} for details. As discussed in section \ref{sec:neuroimaging}, we prefer device that is available to the consumer.

OpenBCI is an open source brain-computer interface that provides access to all the data and algorithms, which makes it good for research, but it is not easy to use for non-specialist. From the more consumer friendly devices, Emotiv EPOC seems to offer good price-quality relationship.

Emotiv EPOC has 16 electrodes, two of which are reference electrodes. Reference electrodes have two different possible locations, P3 and P4 or behind the ears. Other electrodes have fixed locations AF3, AF4, F3, F4, F7, F8, FC5, FC6, P7, P8, T7, T8, O1, O2. See figure \ref{fig:electrode_locations} for illustration. Emotiv EPOC has internal sampling rate of \SI{2048}{Hz} and \gls{ADC} resolution of 16 bit. 

\gls{ADC} converts continuous voltage to the sequence of discrete values or in other words, an analog signal to a digital signal that can be processed by computer.

\newcommand{\patiCHamp}{\tablefootnote{http://www.brainvision.com/files/actiCHamp-PyCorder-Flyer\_US.pdf}}
\newcommand{\pmitsar}{\tablefootnote{http://www.novatecheeg.com/products--software.html}}
\newcommand{\pemotiv}{\tablefootnote{https://emotiv.com/epoc.php}}
\newcommand{\pmindwave}{\tablefootnote{http://store.neurosky.com/products/mindwave-1}}
\newcommand{\mitsarspec}{\tablefootnote{http://www.mitsar-medical.com/eeg-machine/eeg-amplifier-compare/}}
\newcommand{\popenbci}{\tablefootnote{http://openbci.myshopify.com/products/openbci-8-bit-board-kit}}

\begin{table}[h]
	\centering
	\begin{tabular}{|c|c|c|c|c|}\hline
								& Price						& Channels	& Sampling rate	& \gls{ADC} resolution	\\\hline
		Mindwave\pmindwave		& \SI{80}[\$]				& 1			& \SI{512}{Hz}	& 12 bit				\\\hline
		Emotiv EPOC\pemotiv		& \SI{400}[\$]				& 14+2		& \SI{128}{Hz}	& 16 bit				\\\hline
		OpenBCI\popenbci		& \SI{450}[\$]				& 8			& adjustable	& 24 bit				\\\hline
		Mitsar 202\mitsarspec	& \SI{10500}[\$]\pmitsar	& 31+1		& \SI{2}{kHz}	& 24 bit				\\\hline
		atiCHamp\patiCHamp		& \SI{77100}[\$]			& 160		& \SI{25}{kHZ}	& 24 bit				\\\hline
	\end{tabular}
	\caption{Comparison of EEG devices}
	\label{tab:EEG}
\end{table}

\begin{figure}[h]
	\centering
	\begin{subfigure}{0.48\textwidth}
		\includegraphics[width=\textwidth]{electrode_locations.png}
		\caption{Electrode locations used by Emotiv EPOC\protect\footnotemark}
		\label{fig:electrode_locations}
	\end{subfigure}
	\caption{Caption}
	\label{fig:label}
\end{figure}
\footnotetext{http://emotiv.wikia.com/wiki/Emotiv\_EPOC}


\chapter{Brain-Computer Interface}
\section{SSVEP}
\section{c-VEP}
\chapter{Target Identification Methods}
\section{Power spectral density analysis}
Power Spectral density (PSD) analysis is widely used in SSVEP BCIs\cite{bin2009cca}.
\section{Canonical correlation analysis}
Canonical correlation analysis was first introduced by Harold Hotelling in 19xx . Testing citing\cite{scipy}\cite{scikit-learn}\cite{psychopy}

\chapter{SSVEP-based BCI for controlling a robot}
\label{sec:SSVEP_BCI}

The aim of this chapter is to describe the \gls{SSVEP}-based \gls{BCI} designed \dots. The \gls{BCI} is written in python 2.7 and the code is accessible from Github repository. See appendix~\ref{sec:bci_manual} for installation and usage instructions. The visual stimuli are designed using PsychoPy~\cite{psychopy}. \gls{CCA} algorithm is calculated using Scikit-Learn module~\cite{scikit-learn}. Most of other algorithms including \gls{FFT} are calculated using Scipy and Numpy modules~\cite{scipy}. The \gls{BCI} requires only Emotiv EPOC headset and a laptop computer, no specific hardware like digital signal processors or \glspl{LED} are used.

\section{Signal pipeline}
asd
\section{Target identification}
asd
\section{The robot}
asd

\chapter{Results}

As discussed in chapter~\ref{sec:SSVEP_BCI}, the \gls{BCI} written as a practical part of this thesis has many parameters that can be changed and finding the best combination of different parameters needs thorough testing. Unfortunately, thorough testing did not fit into the scope of this thesis. The testing was performed on only one subject. By using trial and error method, settings that are shown in appendix~\ref{sec:parameters} were fixed for further testing.

After fixing the signal pipeline parameters as shown in table~\ref{tab:pipelines}, different target identification parameters were tested. This testing was performed in trials, each trial lasted 60 seconds. Four targets were used with frequencies of 5.45, 6.0, 6.67 and 7.5 Hz. The rest of the target parameters can be seen in table~\ref{tab:targets}. The application chose randomly a target that the subject had to look at and after this randomly chosen target was detected from the \gls{EEG} recording, the application chose randomly another \gls{target} that the subject had to look at and so on. New random \gls{target} was chosen only if the previous had been detected. This continued until 60 seconds had passed. If a detected \gls{target} was not the \gls{target} that the application randomly chose, then the result was classified as false positive. If a detected \gls{target} was the same as the \gls{target} that application randomly chose, then the result was classified as true positive. Accuracy was calculated by dividing the number of true positives with the total number of results.

The application gave the subject visual feedback about which \gls{target} had been detected from the \gls{EEG} recording and showed the current \gls{target} that the subject has to look at as shown in figure~\ref{fig:test_targets}. Laptop computer with \SI{60}{Hz} \gls{refresh rate} 14.0" diagonal LED-backlit HD 16:9 widescreen (1366 x 768) LCD monitor was used for displaying the \glspl{target}. For recording the brain activity, Emotiv EPOC \gls{EEG} device was used.

\begin{figure}[h]
	\centering
	\includegraphics[width=0.6\textwidth]{test_targets.png}
	\caption{The targets used for testing the application and visual feedback for the subject. White triangles show the current target that the subject has to look and the green triangles show the target that was detected.}
	\label{fig:test_targets}
\end{figure}

The \gls{feature extraction} method's weight parameters were fixed as shown in table~\ref{tab:weights}. The method ID corresponds to the method ID in table~\ref{tab:pipelines}. Finally, different \gls{target} identification methods were tested and the results can be seen in figure~\ref{tab:results}. The \gls{target} detection time was calculated by dividing 60 seconds with the number of results in the given trial. The \gls{ITR} was calculated as shown in equation~\ref{eq:itr}.

	\begin{table}[h]
		\centering
		\begin{tabular}{|l|l|l|l|}\hline
Method	& Sensor 	& Harmonic		& Weight\\
ID		& 		 	&  				& 		\\\hline
1		& Sum of all& 1				& 6		\\\cline{2-4}
		& Sum of all& Sum of all	& 6		\\\hline
2		& All	 	& All			& 12	\\\hline
3		& O1	 	& 1				& 1		\\\cline{2-4}
		& O2	 	& 1				& 1		\\\cline{2-4}
		& O1	 	& Sum of all	& 1		\\\cline{2-4}
		& O2	 	& Sum of all	& 1		\\\cline{2-4}
		& Sum of all& Sum of all	& 1		\\\cline{2-4}
		& Sum of all& Sum of all	& 1		\\\hline
4		& All		& All			& 6		\\\hline
		\end{tabular}
		\caption{Weights of the feature extraction methods.}
		\label{tab:weights}
	\end{table}

The threshold, $k$, $m$ and $n$ in table~\ref{tab:results} are the \gls{target} identification parameters that were discussed in section~\ref{sec:identification}.

\begin{table}[h]
	\centering
	\begin{tabular}{|c|c|c|c|c|c|c|c|}\hline
Trial ID	& Target detection 	& Accuracy	& ITR (bits/min)	& Threshold&$k$ & $m$& $n$\\
			& time (sec)		& (\%)		&					&   &  &  & \\\hline
1			& 3.00				& 95.00		& 32.69				& 65& 3& 4& 3\\\hline
2			& 2.86				& 90.48		& 29.30				& 70& 3& 3& 2\\\hline
3			& 2.86				& 80.95		& 20.91				& 65& 3& 5& 3\\\hline
4			& 2.86				& 80.95		& 20.91				& 65& 3& 7& 5\\\hline
5			& 2.73				& 95.45 	& 36.55				& 65& 3& 5& 4\\\hline
6			& 2.73				& 90.91		& 31.16				& 65& 3& 4& 3\\\hline
7			& 2.73				& 90.91		& 31.16				& 65& 3& 6& 4\\\hline
8 			& 2.73	 			& 81.82 	& 22.61 			& 65& 3& 8& 5\\\hline
9 			& 2.40 				& 88.00 	& 32.01 			& 65& 3& 7& 4\\\hline
10 			& 2.40 				& 80.00 	& 24.03 			& 65& 3& 8& 4\\\hline
11 			& 2.31				& 84.62 	& 29.56 			& 65& 3& 3& 2\\\hline
12 			& 2.22 				& 81.48 	& 27.41 			& 60& 3& 4& 3\\\hline
13 			& 2.14 				& 75.00		& 22.19 			& 65& 3& 4& 2\\\hline
Average:	& $2.61\pm 0.28$ 	& $85.81\pm 6.39$	& $27.73\pm 5.11$\\\cline{1-4}
	\end{tabular}
	\caption{Results of 13 trials of testing the application. Threshold is the value that the sum of the weights of the $k$ previous results has to exceed for the corresponding target to be added to the list of results. From the list of results, a target is chosen if there are at least $n$ occurrences of the targets. The list of results has the length of $m$.}
	\label{tab:results}
\end{table}

The results presented in this section can be compared to other \glspl{BCI} that also use Emotiv EPOC headset. The results of the previous \glspl{BCI} can be seen in table~\ref{tab:emotiv_BCIs}.

\addcontentsline{toc}{chapter}{Conclusion}

\chapter*{Conclusion}

The aim of this thesis has been to describe an \gls{SSVEP}-based \gls{BCI} implemented as a practical part of this thesis. In the first chapter the biological background was discussed to understand how brain activity can be measured and a low-cost \gls{EEG} device called Emotiv EPOC was chosen for recording the brain activity.

The second chapter provided and overview of the mathematical and technological background. Widely used \gls{feature extraction} methods called \gls{PSDA} and \gls{CCA} were discussed in this chapter along with signal processing techniques and 

Finally, the third chapter described the implemented application itself. The strengths of the application compared to the existing applications are that this application is open-source, it allows different options for signal pipeline to be easily tested and it combines different \gls{SSVEP} \gls{feature extraction} methods. Although there are hybrid \glspl{BCI} that use different neuromechanisms and therefore different \gls{feature extraction} methods, to the best of the author's knowledge there are no \glspl{BCI} that use different \gls{feature extraction} methods with one neuromechanism, in this case \gls{SSVEP}.

The application could be further developed to support different \gls{EEG} devices and more \gls{feature extraction} methods to find the best combination of different \gls{feature extraction} methods. Using multiple methods at the same time clearly improves the performance of a \gls{BCI}, but on the other hand it is computationally more expensive. But as the technology improves, this becomes less relevant.

[TODO: after measuring the performance of my application, add a paragraph about the that somewhere]



% References
% styles: https://www.sharelatex.com/learn/Bibtex_bibliography_styles

\bibliographystyle{abbrv}
\addcontentsline{toc}{chapter}{References}
\bibliography{thesis}
Internet URLs were valid on ...

% Appendices
% Can be replaced by \appendix

\begin{appendices}
\setcounter{table}{0}
\renewcommand{\thetable}{A\arabic{table}}
%\titleformat{\chapter}{\large\bfseries}{\appendixname~\thesection .}{0.5em}{}
%\titleformat{\chapter}{\normalfont\LARGE\bfseries}{\appendixname~\thechapter}{20pt}{}{}
\renewcommand{\thesection}{\Roman{section}}
\section{Glossary}
\printglossary
\newpage
\section{Acronyms}
\printglossary[type=\acronymtype]
\newpage
\begin{landscape}
\section{Fixed parameters for testing}
\label{sec:parameters}
	\begin{table}[h]
		\centering
		\begin{tabular}{|l|l|l|l|l|l|l|l|l|l|l|}\hline
Method& Extraction& Length& Step	& Sensors		& Detrend	& Detrend	 & Window& Kaiser& Interpolation& Filter\\
ID	& method	& (sec)	& (sec)	& 				& 			& breakpoints& 		 & beta	 & 				&  \\\hline
1	& Sum PSDA	& 2		& 0.125	& O1, O2		& Linear	& 5			 & Kaiser& 14	 & Quadratic	&- \\\hline
2	& CCA		& 2		& 0.125	& O1, O2, P7, P8& Linear	& 5			 & -	 & -	 & -			&- \\\hline
3	& PSDA		& 1		& 0.125	& O1, O2		& Linear	& 3			 & Kaiser& 14	 & Quadratic	&- \\\hline
4	& CCA		& 1		& 0.125	& O1, O2, P7, P8& Linear	& 3			 & -	 & -	 & -			&- \\\hline
		\end{tabular}
		\caption{Settings of the feature extraction method's signal pipelines.}
		\label{tab:pipelines}
	\end{table}
	\begin{table}[h]
		\centering
		\begin{tabular}{|l|l|l|l|l|l|l|l|l|l|}\hline
Target	& Frequency & Harmonics		& Position x& Position y& Color	& Type			& Shape	 & Width 	& Height\\
ID		& (Hz)		& in extraction	& (pixel)	& (pixel)	& 		& 		 		& 		 & (pixel)	&(pixel)\\\hline
1		& 5.45		& 1, 2, 3		& 0		 	& -200		& White	& Single graphic& Square & 150	 	& 150\\\hline
2		& 6.0		& 1, 2, 3		& -300	 	& 0			& White	& Single graphic& Square & 150	 	& 150\\\hline
3		& 6.67		& 1, 2, 3		& 0		 	& 200		& White	& Single graphic& Square & 150	 	& 150\\\hline
4		& 7.5		& 1, 2, 3		& 300	 	& 0			& White	& Single graphic& Square & 150	 	& 150\\\hline
		\end{tabular}
		\caption{Settings of the targets.}
		\label{tab:targets}
	\end{table}
\end{landscape}
%\section{List of figures}
%\makeatletter
%\@starttoc{lof}% Print List of Figures
%\makeatother
%\newpage
%\listoftables

\end{appendices}


\end{document}